\documentclass[letterpaper,12pt]{article}
\synctex=1

\usepackage{todonotes}

\begin{document}
  \section{}
  I find that most of the criticisms the paper makes about existing
  organizations are still true. For example, I have noticed from personal
  experience that switching the group that I am working in to work on a
  different project has more friction than working on a different project with
  the same group of people. I find that the paper's remarks that organizations
  in their current state tend to empasize ``achievement motivation'' to be true
  as well, with most people working towards themselves getting a promotion, as
  opposed to the focus being on a group of people rising the ranks in the
  company together to be still true today.  Even in the workplace today, in my
  recent co-op experience, while efforts are being made to work in groups,
  (like having multiple Agile teams work in their groups) it feels like it is
  still largely an afterthought, and the organizations were not designed from
  the ground up with groups being the smallest unit.

  %The seven arguments the paper
  %makes for why groups are worth considering as ``fundamental building blocks''
  %are all good points in my opinion.

  \todo{not done yet}

  \section{}
  I would consider school in general to mostly be a ``highly individualistic''
  organization, which has been a part of my life.
  A lot of the time, I feel that a ``groupy'' organizations are more effective
  for accomplishing goals such as actually understanding the content. When
  working together, group mates often have knowledge gaps in different parts of
  the content, so everyone in the group can get a better understanding of the
  content by sharing the parts they know well.

  \section{}
  In my life, there have been times when school projects were set up as
  ``groupy'' organizations.  Most of the time, they work out great, however,
  there are some cases where a ``highly individualistic'' organization  might
  have worked better for finishing the project. There are times when work
  cannot be done in parallel due to the nature of the project, and team members
  end up waiting on each other for dependencies to be completed. Furthermore,
  once a portion of the project is done, knowledge must be transferred from one
  group member to the other, which is a good opportunity for error to be
  introduced (Like the telephone game where a sentence is said to one person,
  and the message is passed around. At the end of the activity, the
  participants usually find that the message has changed from what it
  originally was). In situations like these, I would argue that an
  individualistic organization would be more ideal, because one person can keep
  track of what needs to be done, and can always be working on something, as
  opposed to a group where one person is working on a particular task, while
  the rest of the members have to wait. 


  \section{}
  I think that in general, the gender effects metioned in the paper probably
  still hold true today in Canadian society. One may not need to look far to
  find an exception (i.e.\ a woman who socialized into a competitive attitude),
  I would speculate that the ratio of men to women likely changed in recent
  times such that less women socialize into affiliative and relational
  attitudes and more women socialize into more competitive attitudes compared
  to the past. However, I think that men are overall more likely to socialize
  into competitive attitudes than women at this time of writing.

  This trend may be true in my social circle.  When choosing a socially
  distanced activity to do on-line, people who identify as female in our group
  tend to prefer games that are co-operative and require working together as a
  team to accomplish a goal (e.g. Overcooked, Among Us), while most who
  identify as male tend to prefer more competitive games with a highly
  individualistic ranking system (e.g. Counter-Strike: Global Offensive, Call
  of Duty Modern Warfare). It should be noted however, that the number of
  number of women in the social circle may not be statistically significant
  enough compared to the number of men for drawing conclusions such as this. 

\end{document}
