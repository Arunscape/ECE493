\documentclass[letterpaper,12pt]{article}
\synctex=1

\usepackage{hyperref}

\title{ECE 493 Prelab 1}
\author{Arun Woosaree\\
\input{~/student_id}}

\begin{document}
\maketitle
  \section{}
  I find that most of the criticisms the paper makes about existing
  organizations are still true. For example, I have noticed from personal
  experience that switching the group that I am working in to work on a
  different project has more friction than working on a different project with
  the same group of people. I find that the paper's remarks that organizations
  in their current state tend to empasize ``achievement motivation'' to be true
  as well, with most people working towards themselves getting a promotion, as
  opposed to the focus being on a group of people rising the ranks in the
  company together.  Even in the workplace today, in my recent co-op
  experience, while efforts are being made to work in groups, (like having
  multiple Agile teams work in their groups) it feels like it is still largely
  an afterthought, and the organizations were not designed from the ground up
  with groups being the smallest unit. As a result, most organizations today
  seem to be missing out on the seven benefits outlined in the paper which
  argue in favour of ``groupy'' organizations.

  %The seven arguments the paper
  %makes for why groups are worth considering as ``fundamental building blocks''
  %are all good points in my opinion.

  \section{}
  I would consider school in general to mostly be a ``highly individualistic''
  organization, which has been a part of my life.  A lot of the time, I feel
  that ``groupy'' organizations are more effective for accomplishing goals such
  as actually understanding the content.  For example, team members can work on
  tasks in parallel to reach their goals faster. Also, people often have
  knowledge gaps in different parts of the content. Those working in a group
  can get a better understanding of the content by sharing the parts they know
  well, helping to fill in the gaps for other group members. Often times this
  may save time for an instructor as well, since group mates can answer each
  others' questions as opposed to having many individuals ask the instructor
  questions that they might have, usually with multiple duplicates.

  \section{}
  In my life, there have been times when school projects were set up as
  ``groupy'' organizations.  Most of the time, they work out great, however,
  there are some cases where a ``highly individualistic'' organization  might
  have worked better for finishing the project. There are times when work
  cannot be done in parallel due to the nature of the project, and team members
  end up waiting on each other for dependencies to be completed. Furthermore,
  once a portion of the project is done, knowledge must be transferred from one
  group member to the other, which is a good opportunity for error to be
  introduced (Like the telephone game where a sentence is said to one person,
  and the message is passed around. At the end of the activity, the
  participants usually find that the message has changed from what it
  originally was). In situations like these, I would argue that an
  individualistic organization would be more ideal, because one person can keep
  track of what needs to be done, and can always be working on something, as
  opposed to a group where one person is working on a particular task while
  the rest of the members have to wait. 


  \section{}
  I think that in general, the gender effects metioned in the paper probably
  still hold true today in Canadian society. One may not need to look far to
  find an exception (i.e.\ a woman who socialized into a competitive attitude),
  I would speculate that the ratio of men to women likely changed in recent
  times such that less women socialize into affiliative and relational
  attitudes and more women socialize into more competitive attitudes compared
  to the past, however, I think that men are overall more likely to socialize
  into competitive attitudes than women at this time of writing.

  This trend may be true in my social circle.  When choosing a socially
  distanced activity to do on-line, people who identify as female in our group
  seem to prefer games that are co-operative and require working together as a
  team to accomplish a goal (e.g. Overcooked, Among Us), while most who
  identify as male seem to prefer more competitive games with a highly
  individualistic ranking system (e.g. Counter-Strike: Global Offensive, Call
  of Duty Modern Warfare). It should be noted however, that the number of
  number of women in the social circle may not be statistically significant
  enough compared to the number of men for drawing conclusions such as this.

  \setcounter{section}{0}
  \section{}
  The difference seems to be that for small group research, social psychology has an emphasis on
  people and their interactions with other people within the groups, while organizational psychology
  has more focus on how the group as a whole accomplishes tasks, and the technogies used to solve the task.
  Personally, I do not have any experience of my own in doing any sort of psychology research but an
  example of the social type might be studying how the language used to communicate between group members
  evolves over time as they get to know each other, while an example of the organizational type might be
  studying what tools are being used for the communication (like phone calls, or video web conferencing).

  \begin{itemize}
    \item ``context'' in the context of the paper appears to refer to the
environment in which an individual or group is  performing a task in. The paper
mentions how context can influence how a group functions, or in other words, how
a team might work differently in a work environment where there is pressure to
deliver an outcome that might affect someone's career, versus a low risk
research environment with made-up tasks.
    \item ``workflow'' in the context of the paper appears to refer to the
          processes a group might set up to handle tasks, and how they get the tasks
          completed as a group. i.e.\ how tasks go from a state of being incomplete, to a
          completed state.
    \item ``levels'' in the context of the paper appears to refer to where an individual is located within a nested
          organizational structure. Examples of levels include the individual level, the team level, and higher order levels above
          The paper talks about how there are different emergent properties that appear at these different levels, even if
          each individual has their own, distinct thoughts and behaviours
    \item ``time'' in the context of the paper has exactly the same meaning as the ordinary meaning of the word. The paper
          talks about how time plays an important role in the evolution of a team, and how this effect cannot be observed in
          research studies which generally observe teams in a short time span where there is not enough time for teams to mature.
  \end{itemize}

  \section{}
  First, a group size of three is a small group. I have found in my personal
experience that working in a small group is much less complicated than working
in a larger group. I would guess that the instructor sees this trend too.  There
are less interpersonal conflicts, and developing a consenseus is much easier. It
is also quicker to get to know your group mates, since remembering two new names
is easier than remembering ten, for example. It is also common in larger groups
to form smaller groups within, and this might be a situation that has not worked
well in the past for the capstone project. Larger groups are also expected to
complete more work. Another reason to prefer a smaller group over a larger one
is that it is easier to hold everyone accountable in a smaller group, and make
sure everyone is doing their fair share.  Additionally, with a lower number of
people, there are bound to be less silos, where one person might be working hard
on a specific thing, yet the rest of the team has no idea what they are doing.
Everyone should have a general idea of what their mates are doing,  because
there are fewer group mates.  Knowledge transfers also happen much quicker with
the decreased complexity. Three is also an odd number, which helps with decision
making because there are no ties.

\section{}
The paper mentions that various studies have found wildly different results on
whether or not diversity in teams is beneficial. Other factors matter too, like
the nature of the tasks, and if the tasks are menial, or if they require
creativity, or if many different skills are required to do the task versus doing
simple, mostly similar tasks.

According to \url{https://saylordotorg.github.io/text_organizational-behavior-v1.1/s06-02-demographic-diversity.html},
benefits of demographic diversity in teams include higher creativity in decision making, which results in
better decisions to be made overall, due to more outcomes being considered. More ideas are generated, but also
because peoples' backgrounds differ, they also raise more potential issues, which leads to better choices being made overall.
This page makes some bold claims that diversity may result in higher stock prices, lower litigation expenses, and higher overall performance as a company. It also mentions a few drawbacks, such as ``faultlines'', where people may divide themselves
into subgroups with similar backgrounds, and issues that may arise with different ages, religion, and sexual orientations.
The following page \url{https://en.wikipedia.org/wiki/Team_diversity#Demographic_diversity} mentions that
`` individuals who are different from their work team in demographic characteristics are less psychologically committed to their organizations, less satisfied and are therefore more absent from work''. However, it also mentions that age diversity may not be
ideal, since studies have shown that people are more likely to part from the group. Gender and different ethnic backgrounds
tend to have more conflict at the beginning, but these problems also tend to go away once the team matures, and perform better
than a homogeneous team over time. It also mentions that while there are these specific benefits and drawbacks, the impacts
of demographic diversity as a whole are still contested, due to research not keeping up, and also synthetic tests
likely not being representative of an actual workplace, which is an issue mentioned in the original paper.

\section{}
The Big Five personality factors according to \url{https://en.wikipedia.org/wiki/Team_diversity#Demographic_diversity} are:
\begin{enumerate}
  \item openness to experience: people in this category tend to be more willing to try out something new, and also tend to
        value experiences such as an adventure, art, or a particularly moving film. Compared to someone who is less open, they
        might be more creative, but also more curious and more willing to take risks.
  \item conscientiousness: people in this category are very deliberate in their actions. They usually have a great work ethic
        self-discipline, and strive to exceed their goals. These people have a high attention to detail, and tend to prefer their activities to have a concrete plan over spontaneity
  \item extraversion: extraverts tend to enjoy being around with other people, and dislike being alone with their thoughts. They usually are the ones to initiate conversations with others, and enjoy the attention of others. Usually, within a group,
        extraverts are the most visible members, and can appear assertive.
  \item agreeableness:  agreeable people tend to appear friendly, kind, and generous They value getting along with other people, and are usually willing to compromise, even if something goes against their values. They usually have concern for the general
        well-being of others, and are usually willing to set some time aside to help someone out. They also tend to see the good side in people and believe that humans are generally good instead of evil.
  \item neuroticism: people in this category tend to experience bad feelings often, and appear to be emotionally unstable. ``Bad'' feelings include things like anger, anxiety, and depression. These people tend to have a low tolerance for stress, and it does
        not take much to get stressed out, irritated, sad, or to ruin their day in some other way. Inconveniences that may appear minor to some might actually become a huge hurdle to overcove for a person with neuroticism
\end{enumerate}
\end{document}

\section{}
Interviews today are highly based on the individual, which does not take into account at all
how well the individual will work in a team.
Organizations might want to switch their current tactics for
constructing teams. It could be that the future of talent acquisition is organizations evaluating their
candidates while they are interacting in a group, instead of doing a classic one-on-one style of interview. Whether the person is being formally assigned to a group or not, informal groups are bound to be formed in the organization, and the individual will eventually have to interact with others, and complete tasks in groups. More reasearch might be done so that
we can find out what characteristics are required to make a better team, and once identified, these could be used to
select good candidates, and get a good team together from the start, methodically, instead of by trial and error.


\section{}
The four stages are:
\begin{enumerate}
  \item forming:
        in the forming stage, the group meets for the first time as the group is formed. They discuss what needs to be done,
        but they also discuss the challenges they may face. The main focus is on defining team norms, and what is expected
        from each member. Team members might begin their tasks, but the work tends to be done independently.
  \item storming:
        in the storming phase, people who
        might have been too anxious to share a controversial opinion or concern might feel ready to voice their concerns.
        Disagreements may happen, hence the ``storming'' name. The team may have to revisit their original goals and tweak them,
        or break down their tasks into more digestible chunks. This is about getting back on track, and sorting out interpersonal conflicts.
  \item norming:
        after the storming phase, things tend to settle down in the norming phase. Team members start to get to know each other better, and they should trust each other more. People should start to feel more comfortable with working in the group. Team members start to become more cooperative with each other,
        and conflict is generally avoided yet there is still tolerance for constructive criticism. Team members might want to work on tasks together. There is a focus on productivity at this stage
  \item performing:
        in the performing stage, team members should be mostly autonomous, and the group as a whole
        should be performing at its peak. At this point, everyone is mostly comfortable with each other,
        and everyone knows what needs to be done to reach their goals. The co-operative spirit is continued from the
        previous stage.
        Most of the progress is made here, or the project done to completion.
        It is possible to regress to an earlier stage. A new scrum master for example may revert the team to the storming phase
        as they adjust to their new expectations, and potential conflicts arise.
\end{enumerate}
reference: \url{https://en.wikipedia.org/wiki/Tuckman\%27s_stages_of_group_development}
