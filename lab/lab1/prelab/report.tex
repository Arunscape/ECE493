\documentclass[letterpaper,12pt]{article}
\synctex=1

\usepackage{todonotes}

\begin{document}
  \section{}
  I find that most of the criticisms the paper makes about existing
  organizations are still true. For example, I have noticed from personal
  experience that switching the group that I am working in to work on a
  different project has more friction than working on a different project with
  the same group of people. I find that the paper's remarks that organizations
  in their current state tend to empasize ``achievement motivation'' to be true
  as well, with most people working towards themselves getting a promotion, as
  opposed to the focus being on a group of people rising the ranks in the
  company together.  Even in the workplace today, in my recent co-op
  experience, while efforts are being made to work in groups, (like having
  multiple Agile teams work in their groups) it feels like it is still largely
  an afterthought, and the organizations were not designed from the ground up
  with groups being the smallest unit. As a result, most organizations today
  seem to be missing out on the seven benefits outlined in the paper which
  argue in favour of ``groupy'' organizations.

  %The seven arguments the paper
  %makes for why groups are worth considering as ``fundamental building blocks''
  %are all good points in my opinion.

  \section{}
  I would consider school in general to mostly be a ``highly individualistic''
  organization, which has been a part of my life.  A lot of the time, I feel
  that ``groupy'' organizations are more effective for accomplishing goals such
  as actually understanding the content.  For example, team members can work on
  tasks in parallel to reach their goals faster. Also, people often have
  knowledge gaps in different parts of the content. Those working in a group
  can get a better understanding of the content by sharing the parts they know
  well, helping to fill in the gaps for other group members. Often times this
  may save time for an instructor as well, since group mates can answer each
  others' questions as opposed to having many individuals ask the instructor
  questions that they might have, usually with multiple duplicates.

  \section{}
  In my life, there have been times when school projects were set up as
  ``groupy'' organizations.  Most of the time, they work out great, however,
  there are some cases where a ``highly individualistic'' organization  might
  have worked better for finishing the project. There are times when work
  cannot be done in parallel due to the nature of the project, and team members
  end up waiting on each other for dependencies to be completed. Furthermore,
  once a portion of the project is done, knowledge must be transferred from one
  group member to the other, which is a good opportunity for error to be
  introduced (Like the telephone game where a sentence is said to one person,
  and the message is passed around. At the end of the activity, the
  participants usually find that the message has changed from what it
  originally was). In situations like these, I would argue that an
  individualistic organization would be more ideal, because one person can keep
  track of what needs to be done, and can always be working on something, as
  opposed to a group where one person is working on a particular task while
  the rest of the members have to wait. 


  \section{}
  I think that in general, the gender effects metioned in the paper probably
  still hold true today in Canadian society. One may not need to look far to
  find an exception (i.e.\ a woman who socialized into a competitive attitude),
  I would speculate that the ratio of men to women likely changed in recent
  times such that less women socialize into affiliative and relational
  attitudes and more women socialize into more competitive attitudes compared
  to the past, however, I think that men are overall more likely to socialize
  into competitive attitudes than women at this time of writing.

  This trend may be true in my social circle.  When choosing a socially
  distanced activity to do on-line, people who identify as female in our group
  seem to prefer games that are co-operative and require working together as a
  team to accomplish a goal (e.g. Overcooked, Among Us), while most who
  identify as male seem to prefer more competitive games with a highly
  individualistic ranking system (e.g. Counter-Strike: Global Offensive, Call
  of Duty Modern Warfare). It should be noted however, that the number of
  number of women in the social circle may not be statistically significant
  enough compared to the number of men for drawing conclusions such as this.

  \setcounter{section}{0}
  \section{}
  The difference seems to be that for small group research, social psychology has an emphasis on
  people and their interactions with other people within the groups, while organizational psychology
  has more focus on how the group as a whole accomplishes tasks, and the technogies used to solve the task.
  Personally, I do not have any experience of my own in doing any sort of psychology research but an
  example of the social type might be studying how the language used to communicate between group members
  evolves over time as they get to know each other, while an example of the organizational type might be
  studying what tools are being used for the communication (like phone calls, or video web conferencing).

  \begin{itemize}
    \item ``context'' in the context of the paper appears to refer to the
environment in which an individual or group is  performing a task in. The paper
mentions how context can influence how a group functions, or in other words, how
a team might work differently in a work environment where there is pressure to
deliver an outcome that might affect someone's career, versus a low risk
research environment with made-up tasks.
    \item ``workflow'' in the context of the paper appears to refer to the
          processes a group might set up to handle tasks, and how they get the tasks
          completed as a group. i.e.\ how tasks go from a state of being incomplete, to a
          completed state.
    \item ``levels'' in the context of the paper appears to refer to where an individual is located within a nested
          organizational structure. Examples of levels include the individual level, the team level, and higher order levels above
          The paper talks about how there are different emergent properties that appear at these different levels, even if
          each individual has their own, distinct thoughts and behaviours
    \item ``time'' in the context of the paper has exactly the same meaning as the ordinary meaning of the word. The paper
          talks about how time plays an important role in the evolution of a team, and how this effect cannot be observed in
          research studies which generally observe teams in a short time span where there is not enough time for teams to mature.
  \end{itemize}

  \section{}
  First, a group size of three is a small group. I have found in my personal
experience that working in a small group is much less complicated than working
in a larger group. I would guess that the instructor sees this trend too.  There
are less interpersonal conflicts, and developing a consenseus is much easier. It
is also quicker to get to know your group mates, since remembering two new names
is easier than remembering ten, for example. It is also common in larger groups
to form smaller groups within, and this might be a situation that has not worked
well in the past for the capstone project. Larger groups are also expected to
complete more work. Another reason to prefer a smaller group over a larger one
is that it is easier to hold everyone accountable in a smaller group, and make
sure everyone is doing their fair share.  Additionally, with a lower number of
people, there are bound to be less silos, where one person might be working hard
on a specific thing, yet the rest of the team has no idea what they are doing.
Everyone should have a general idea of what their mates are doing,  because
there are fewer group mates.  Knowledge transfers also happen much quicker with
the decreased complexity. Three is also an odd number, which helps with decision
making because there are no ties.


\end{document}
